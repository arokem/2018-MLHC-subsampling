\begin{abstract}

Deep learning has tremendous potential utility in the
classification of biomedical images. For example, images acquired with retinal
optical coherence tomography (OCT) can be used to classify patients
with adult macular degeneration (AMD), and distinguish them from healthy
control patients. Previous research suggests that large amounts of data are
required in order to train deep learning (DL) algorithms, because of
the large number of parameters that need to be fit. To study the data
requirements of biomedical DL, we applied a subsampling procedure to 
classification of AMD patients from OCT data. We found that performance 
decreases approximately constantly with each halving of the data. These 
results suggest that deep learning algorithms can be trained on
relatively moderate amounts of data, provided that images are homogenous, and 
the effective number of parameters is sufficiently small. Furthermore, we 
demonstrate that in this application, performance evaluation with a separate test set 
that is not used in any part of the training does not differ substantially from 
performance evaluation with a validation data-set that was also used to 
determine the optimal stopping point for training.

\end{abstract}
