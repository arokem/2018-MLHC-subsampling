\section{Methods}

This study was approved by the Institutional Review Board of
the University
% of Washington (UW)
and adhered to the tenets of the Declaration of Helsinki and the Health
Insurance Portability and Accountability Act.

\subsection{Data extraction}
Macular OCT scans were acquired in the course of clinical care, using a
Heidelberg Spectralis OCT scanner (Heidelberg Engineering, Heidelberg, Germany).
High-resolution images of the retinal cross-section were obtained using a
61-line raster scan. All of the images from the period 2006 to 2016 were
extracted using an automated extraction tool from the instrument imaging
database. The images were linked by patient medical record number and dates to
the clinical data stored in EPIC. Specifically, all clinical diagnoses and the
dates of every clinical encounter, macular laser procedure, and intravitreal
injection were extracted from the EPIC Clarity tables.

\subsubsection{Patient and Image Selection}

A normal patient was defined as having no retinal International Classification
of Diseases, 9th Revision (ICD-9) diagnosis and better than 20/30 vision in both
eyes during the entirety of their recorded clinical history at the University. An AMD
patient was defined as having an ICD-9 diagnosis of AMD (codes 362.50, 362.51,f
and 362.52) by a retina specialist, at least 1 intravitreal injection in either
eye, and worse than 20/30 vision in the better-seeing eye. Patients with other
macular pathology by ICD-9 code were excluded. These parameters were chosen
\emph{a priori} to ensure that macular pathology was most likely present in both
eyes in the AMD patients and absent in both eyes in the normal patients.
Consecutive images of patients meeting these criteria were included, and no
images were excluded due to image quality. Labels from the EMR were then linked
to the OCT macular images, and the data were stripped of all protected health
identifiers.

As most of the macular pathology is concentrated in the foveal region, the
decision was made \emph{a priori} to select the central 11 images from each
macular OCT set, and each image was then treated independently, and labeled as
either normal or AMD. The images were histogram equalized and the resolution
down-sampled to 192 by 124 pixels to accommodate RAM limitations.

\subsection{Deep Learning Classification Model}

A modified version of the VGG16 convolutional neural network
\cite{Simonyan2014-al} was implemented using Caffe \cite{jia2014caffe}. This
network was originally designed to classify categories in natural images and was
adapted here to classify healthy and AMD retinae. Weights were initialized using
the Xavier algorithm \cite{Glorot2010-is}. Training was then performed using
multiple iterations, each with a batch size of 100 images. ADAM optimization
\cite{Kingma2014-jl} was used with a starting learning rate of $2 \times
10^{-7}$ and the momentum parameters set to 0.9 and 0.99. The loss of the model
was recorded at each training iteration, and cross-validation with a separate
validation set was conducted every 250 iterations. The training was stopped when
the loss of the model decreased and the accuracy of the validation set
also decreased (indicating that the model was in the over-fitting regime).

\subsection{Sub-sampling experiments}

At the outset of the experiments, a random subset of 10\% of the images were
segregated at the patient level into a separate \emph{test set} of images. These
would be used to test the performance of the DL network at the end of training.
The remaining 90\% of the images were then segregated into 11 replicates of
random subsets of 4\%, 8\%, 16\%, 32\%, 64\%, and 100\% of the available images.
Within each subset, the images were again subdivided into 75\% for training and
25\% for validation. Care was taken to ensure that the validation set and the
training set contained images from a mutually exclusive group of patients (ie,
no single patient contributed images to both the training and validation sets).
The order of images of the training set was then randomized in each replication
condition.

Each replication condition was then trained for a total of 75,000 iterations
and the maximal validation accuracy was recorded. The weights at the time of
the maximal validation accuracy was used to assess the performance of the
network against the held-out test set.
